%==========================================================================
\section{Hybrid type theory}

\subsection{Montague grammar and nominal intensional logic}

The following definitions follow Montague's treatment; the only deviations
from his original work are the clauses we have added for handling nominals
and the introduction of the $@$ operator...

As usual the set $\mathsf{TYPES}$ is defined recursively 
\begin{equation*}
\mathsf{TYPES}::=t\mid e\mid \langle a,b\rangle
\end{equation*}

The set $\mathsf{ME}_{a}$ of meaningful expressions of type\textbf{\ }$a$ is
defined recursively as follows:

\begin{enumerate}
\item Every nominal $i$ is in $\mathsf{ME}_{t}$

\item Every constant of type $a$ is in $\mathsf{ME}_{a}$, for any type $a$

\item Every variable of type $a$ is in $\mathsf{ME}_{a}$, for any type $a$

\item If $\alpha \in \mathsf{ME}_{a}$ and $u_{b}$ is a variable of type $b$,
then $\lambda u_{b}\alpha \in \mathsf{ME}_{\langle b,a\rangle }$

\item If $\gamma \in \mathsf{ME}_{\langle b,a\rangle }$ and $\beta \in 
\mathsf{ME}_{b}$ then $\gamma \beta \in \mathsf{ME}_{a}$

\item If $\alpha $ and $\beta $ are both in $\mathsf{ME}_{a}$, then $\alpha
=\beta \in \mathsf{ME}_{t}$

\item If $\varphi $ is in $\mathsf{ME}_{t}$, then $\lnot \varphi $ is also
in $\mathsf{ME}_{t}$

\item If $\varphi $ and $\psi $ are in $\mathsf{ME}_{t}\ $, then $\varphi
\wedge \psi $ is in $\mathsf{ME}_{t}$

\item If $\varphi $ is in $\mathsf{ME}_{t}$ and $u_{a}$ is a variable of any
type $a$, then $\exists u_{a}\varphi $ $\ $is in $\mathsf{ME}_{t}$\textbf{\ }

\item If $\varphi $ is in $\mathsf{ME}_{t}$, then $\Diamond \varphi $ is in $%
\mathsf{ME}_{t}$

\item If $\alpha $ is in $\mathsf{ME}_{a}$, then $@_{i}\alpha \in \mathsf{ME}%
_{a}$
\end{enumerate}

\subsection{Semantics}

The structures used to interpret this powerful language contain a hierarchy
of types plus the usual ingredients of Kripke's structures; namely, a
universe of worlds and the accessibility relation. We also have a function
giving to each constant in the language its appropriate interpretation in the
hierarchy.

A (standard) structure for $\mathcal{HTT}$ is a pair 
$\mathcal{M}=\langle \mathcal{S},\mathsf{F}\rangle $ such that%
$$
\mathcal{S}=\left\langle \langle \mathsf{D}_{a}\rangle _{a\in \mathsf{TYPES}%
},W,R\right\rangle
$$
is a skeleton, where:

\begin{enumerate}
\item $\langle \mathsf{D}_{a}\rangle _{a\in \mathsf{TYPES}}$, the hierarchy
of types, is defined recursively as follows (where $a$ and $b$ are types):%
$$
\begin{array}{rcl}
\mathsf{D}_{e} & = & A\text{ (where }A\not=\varnothing \text{ is the set of
individuals)} \\ 
\mathsf{D}_{t} & = & \text{ is a two element set (the truth values)} \\ 
\mathsf{D}_{\langle a,b\rangle } & = & \mathsf{D}_{b}{}^{\mathsf{D}_{a}}%
\text{ is the set of all functions from }\mathsf{D}_{a}\text{ into }\mathsf{D%
}_{b}%
\end{array}%
$$

\item $W$ is the set of worlds, $W\not=\varnothing $

\item $\mathsf{R}\subseteq \mathsf{W}\times \mathsf{W}$ is the accessibility
relation

\item $\mathsf{F}$ is a function whose domain is $\mathsf{NOM}\cup \mathsf{%
CON}$ of $\mathcal{HTT}$, $\mathsf{F}$ assigns to each nominal a function
from $\mathsf{W}$ to the set of truth values and to each non-logical
constant a function from $\mathsf{W}$ to an element in the hierarchy of
types of the appropriate type 
\begin{equation*}
\mathsf{F}:\mathsf{NOM}\cup \mathsf{CON}\longrightarrow \bigcup\limits_{a\in 
\mathsf{TYPES}}\left( \left. \mathsf{D}_{a}\right. ^{W}\right)
\end{equation*}
\end{enumerate}
