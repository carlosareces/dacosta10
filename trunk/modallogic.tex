
%==========================================================================
\section{The Modal Scale}
\fixme{Start with an general introduction on modal logic}

\fixme{Then  switch to '\textbf{basic} modal logic}

The previous section was all about \emph{ideas}.  We have introduced
the different musical themes (and hinted to the way they relate to 
each other) that we will weave into our composition.  In this section 
we will start our formal work, by introducing the musical scale in 
which the composition will be cast: \emph{Modal Logic}.

The classical perspective on modal logic will tell that they are obtained 
from classical standard logic systems
---propositional logic, first-order logic, second order logic, etc.--- by
adding \emph{``non-truth-functional''}
operators, named modalities; the goal being to use these operators 
to gain the ability to \emph{qualify truth}.

Indeed, classicists would maintain that in modal Logic truth can be qualified. 
That is to say, beyond its descriptive or denotative content, expressions have 
also an auto-reflexive dimension. 
%Modal formalism are able to capture dynamic
%instances, in which truth can be relative. 
Some traditional qualifiers are: necessity, possibility, 
contingency, obligation, belief, knowledge, temporality, etc. In fact, 
if something is shown by this perspective, is that there
is not a unique modal logic but a whole family of them.

From a contemporary perspective, modal logics serve many purpuses.
To start with, and in complete accordance with the classical view, 
modal logics are recognized as marvelous tools for modeling and proving. 
Many notions in linguistic, philosophy, computing and science, etc.\ have a 
\emph{modal} character. Modal logic, thus,
provides a language and semantics to represent these notions in a rigorous
way. They also provide different reasoning mechanisms to obtain deductions or
derivations. Modal languages and their corresponding inference systems has been 
used to model and reason on situations involving \emph{necessity}
and \emph{possibility}, \emph{time}, \emph{beliefs}, \emph{computation}, and
many other. Crucially, all these applications share graph-like structures to represent
their main concepts (networks of possible worlds, flows of time, epistemic
alternatives, computational states, etc.)

More recently, modal logics have been recognized as tools for identifying interesting 
fragments of classical logics (via translation techniques)~\cite{Blackburnetal2007}. 
Modal logics can be seen as fragments of standard logics that inherit its standard semantics
(in terms of relational models) but with restricted expressive power. 
In this way, modal logics can be used to define logical subsystems of well know classical 
languages like first- or higher-order logics, but with specially tailored properties 
such as the decidability property of many modal logics. In
particular, under this view, modal logics are interesting tools to study the
balance between expressiveness and computational complexity in formal systems.

%\subsection{The relational semantics impact}
%
%The best known style of modal semantics is the \emph{relational} or 
%\emph{Kripke} semantics. After studying the class of Kripke models and their
%semantics, it is easy to reach the conclusion that pure modal logic axioms
%capture semantical properties of the accessibility relation $R$, \emph{i.e.},
%modal formulas are valid in models with the corresponding property of $R$:
%
%\begin{itemize}
%\item $T:=\square \varphi \rightarrow \varphi $ is valid, for reflexive $R$
%
%\item $4:=\square \varphi \rightarrow \square \square \varphi \ $ is valid,
%for transitive $R$
%
%\item $D:=\square \varphi \rightarrow \Diamond \varphi $ is valid, for
%serial $R$
%
%\item $B:=\varphi \rightarrow \square \Diamond \varphi $ is valid, for
%symmetric\ $R$
%
%\item $5:=\Diamond \varphi \rightarrow \square \Diamond \varphi $ is valid,
%for euclidean\ $R$
%
%\item $G:=\Diamond \square \varphi \rightarrow \square \Diamond \varphi $ is
%valid, for incestuous\ $R$
%\end{itemize}
%
%In fact, the axioms of a modal logic try to characterize the properties of
%their own accessibility relation. The language is very effective, but not
%always successful. Moreover, we have other alternatives languages. To put it
%succinctly:
%
%\begin{itemize}
%\item Modal models are relational structures and they are extremely common
%in \textit{Classical Model Theory}. 
%
%\item On the other hand, temporal relation is \emph{asymmetric} and
%\emph{irreflexive}, but these are not definable in orthodox tense logic.
%\end{itemize}

\paragraph{Evaluating modal logic}

Modal logics are a counterpart of classical logics. The question is what do
we gain when shifting from classical to modal logic? There are five key
points that make modal logics a very useful tool:

\begin{enumerate}
\item Better understanding, precision and useful axiomatization of the
structures formed by \emph{states, transitions} and \emph{procedures}.

\item Very concise formal expression (operator designated \emph{ad hoc})

\item Modal logics are \emph{decidable}, which is an improved situation in
comparison to the one encountered in FOL or other higher order logics.

\item \emph{Locally} focused: to better understand the \emph{structure of
transitions} between states we place ourselves inside the structure and
travel along

\item We gain an \emph{internal }perspective (AUTOMATON:= visiting the
accessible states)
\end{enumerate}

On the other hand, what is missing when shifting to modal logics?

\begin{enumerate}
\item States are crucial, but we cannot refer to them.

\item The accessibility relation is essential, but it is not explicit in the
language.

\item In tableau calculus metalanguage tools are needed to solve this
shortcoming.
\end{enumerate}

%==========================================================================
%\subsection{Hybrid logic}

\subsection{A Hybrid Melody}

\fixme{relate this to ``point of reference''} Prior~\cite{Prior1967} objected Reichenbach's ideas about the point of
reference. This is ironic, as Areces and Blackburn pointed 
out~\cite{ArecesBlackburn2005} since in that same book~\cite{Prior1967}, Prior introduced
a tool that allows integrating his own ideas with those of Reichenbach in a
very simple way.

This tool's key idea is to add a second sort of propositional symbols ($i$, $%
j$, $k$, etc.), called nominals, to the propositional symbols of orthodox
tense logic ($p$, $q$, $r$, etc.). Nominals allow to formally ``talk'' about
time instants because each nominal symbol is only true at exactly one time in any
model; the time the nominal \emph{``is naming''}. Nominals will be the central
element of \emph{modern hybrid logic}. This is a simple change, but it
immediately produces a more expressive logic, named here \emph{nominal tense
logic}~\cite{Blackburn1994}. Let's see what this means.

Consider the formula of orthodox tense logic
$$
F(r\wedge p)\wedge \ F(r\wedge q)\rightarrow F(p\wedge q)
$$
expressing that if in the future there is a time where both $r$ and $p$ are
true together, and in the future there is a time where $r$ and $q$ are true
together, then there is a time in the future where $p$ and $q$ are true
together. It is clearly not always true because the future times that
witness $p$ and $q$ could be different. Now consider the formula of nominal
tense logic obtained by replacing the propositional symbol $r$ by the nominal
symbol $i$ 
$$
F(i\wedge p)\wedge F(i\wedge q)\rightarrow F(p\wedge q)
$$
This formula is true in every model because the future times witnessing $p$
and witnessing $q$ are both making the nominal $i$ true, and there is only
one time doing this for $i$, by definition. Thus, there is a future time
where $p$ and $q$ are both true, something expressible in nominal tense
logic.

Therefore, nominals allow to express Reichenbach's
points of reference. Moreover, representations in nominal tense improve
Reichenbach's representations~\cite{ArecesBlackburn2005}. As an example,
consider the formula $P(i\wedge P\varphi )$ stating that there is some time
in the past, named with the nominal $i$, and that the event $\varphi $
happened before that past time. It is the representation of the pluperfect
modelization of Reichenbach and combines Reichenbach's insight on the role
played by temporal reference with Prior's insistence on the privileged role
of tensed talk. Figure~\ref{fig2} shows a modification of the table given earlier,
obtained by adding nominal for tense logical representations. In some cases
the nominal tense logical representations improve Reichenbach's. For
instance, the future-in-the-past now has a single representation: the
formula $P(i\wedge F\varphi )$ expresses that there is a reference time $i$
in the past, and that the point of event occurs to the future of $i$, which
is what the future-in-the-past means. With this representation, it is not
necessary to use irrelevant different positions of the point of reference
with respect to the point of speech, as Reichenbach did.

\begin{figure}[h]
\begin{center}
\begin{tabular}{|l|l|l|}
\hline
\textbf{Structure} & Representation & \emph{Example} \\ \hline
\textbf{E-R-S} & $\left\langle P\right\rangle (i\wedge \left\langle
P\right\rangle \varphi )$ & \emph{Alba had sang} \\ \hline
\textbf{E,R-S} & $\left\langle P\right\rangle (i\wedge \varphi )$ & \emph{%
Alba sang} \\ \hline
\textbf{R-E-S }or\textbf{\ R-S,E }or \textbf{R-S-E} & $\left\langle
P\right\rangle (i\wedge \left\langle F\right\rangle \varphi )$ & \emph{%
Alba would sing} \\ \hline
\textbf{E-S,R} & $i\wedge \left\langle P\right\rangle \varphi $ & \emph{%
Alba has sung} \\ \hline
\textbf{S,R,E} & $i\wedge \varphi $ & \emph{Alba sings} \\ \hline
\textbf{S,R-E} & $i\wedge \left\langle F\right\rangle \varphi $ & \emph{%
Alba is going to sing} \\ \hline
\textbf{S-E-R} or \textbf{S,E-R }or\textbf{\ E-S-R} & $\left\langle
F\right\rangle (i\wedge \left\langle P\right\rangle \varphi )$ & \emph{%
Alba will have sung} \\ \hline
\textbf{S-R,E} & $\left\langle F\right\rangle (i\wedge \varphi )$ & \emph{%
Alba will sing} \\ \hline
\textbf{S-R-E} & $\left\langle F\right\rangle (i\wedge \left\langle
F\right\rangle \varphi )$ & \textquestiondown ? \\ \hline
\end{tabular}%
\end{center}

\caption{Reichenbach`s referential analysis of tense using nominals}\label{fig2}
\end{figure}

In nominal tense logic we can find the following characteristic elements of
hybrid modern logic:

\begin{itemize}
\item \textbf{Nominals}. Prior invented a representation for referring to
instants of time: \emph{sorting the propositional symbols and using terms
as formulas} 
$$
\ATOM \cup \NOM
$$
The atomic formulas $i\in \NOM$ received the expected
interpretation, just a singleton.

\item \textbf{Universal modality}. The diamond form of the universal
modality is $\mathbf{E}$\fixme{not mentioned before}

\item \textbf{Quantifiers}. Prior added the quantifiers $\forall$ and 
$\exists$ allowing then to bind nominals
\end{itemize}

This is neat. However, as we pointed out at the start of the paper, the
discussion has been conducted within the confines of propositional nominal
tense logic. If we want to apply these ideas to real grammars for natural
language, that is not good enough. And this leads us to the main topic of
the paper: adding nominals to Montague's IL.

\subsection{Counterpoint}

\emph{States} are crucial in modal semantics but we cannot refer to them with the formal logic. But we can in hybrid logic!. In fact in hybrid logic
we enhance expressivity of modal logic:

\begin{itemize}
\item To introduce explicit references to the elements of the model domain;
for instance, states, days, years, etc.

\item To express that two elements of the domain are related; \emph{i.e.,}
a moment precedes another

\item To improve expressive power by modeling time indexicals; as an
example, yesterday, today, tomorrow, now, etc.

\item To have an elegant proof theory, closed to labelled deduction systems

\item To define properties of frames which are relevant to temporal logic
(irreflexivity, asymmetry)
\end{itemize}

Hybrid logic builds on modal simplicity whilst adding nominals and the $@$
operator to improve its expressive power. 
Nominals refer to individual nodes in the Kripke structure and the $@$
operator allows us to state in the object language that a formula is true at
a given point, while the accessibility relation becomes explicit.

As already noted, in a hybrid language we have two sorts of atomic formulas;
namely $\ATOM \cup \NOM$. We add a new set of modal
operators $\{ @_{i}\mid i\in \NOM\}$ and we form new
formulas in this extended language: $\NOM \subseteq \FORM$ FF
and $@_{i}\varphi \in \FORM$.

In the hybrid semantic we still have Kripke models and we add interpretation
of new formulas:
$$
\mathcal{A},w\Vdash i~\ \text{iff\ }~\text{the instant }w\text{ is named\ }i%
\text{ iff \ }i^{\mathcal{A}}=\left\{ w\right\}
$$
while the interpretation of the new operator $@$ reads%
$$
\mathcal{A},w\Vdash @_{i}\varphi ~\ \text{iff\ }~\mathcal{A},v\Vdash \varphi
$$
$v$ being the unique element of $W$ where $i$ is true, namely 
$i^{\mathcal{A}}=\{v\}$.

\emph{Reflexivity}, \emph{symmetry} and \emph{transitivity} of identity are
now \fixme{pure formulas not introduced before} pure formulas:%
$$
@_{i}i\qquad @_{i}j\rightarrow @_{j}i\qquad @_{i}j\wedge @_{j}k\rightarrow
@_{i}k
$$
And we can also express that two points are accessible: $@_{s}\Diamond t\ $%
says that states named by $s\ $and\ $t$ are related by accessibility\ $R$

\emph{Reflexivity}, \emph{symmetry} and \emph{transitivity} of accessibility
are pure formulas as well:%
$$
@_{i}\Diamond i\qquad @_{i}\square \Diamond i\qquad \Diamond \Diamond
i\rightarrow \Diamond i
$$

\emph{Irreflexivity}, \emph{asymmetry}, \emph{antisymmetry} and\emph{\
intransitivity} of accessibility are also pure formulas: 
$$
@_{i}\lnot \Diamond i\qquad @_{i}\lnot \Diamond \Diamond i\qquad
@_{i}\square (\Diamond i\rightarrow i)\qquad \Diamond \Diamond i\rightarrow
\lnot \Diamond i
$$


\paragraph{Zen Philosophy.}
As an example of the expressive power of hybrid logic let us see the
following.

In \emph{The book of perfect emptiness} Tang de Ying asked Xia Ge: 
\emph{``Did things exist at the dawn of time?''}.

Xia Ge answered: \emph{``If things had not existed at the
dawn of time, how could they possibly exist today? By the same token, men in
the future could believe that things did not exist today.''}

This argument can be reformulated in this way.

\begin{itemize}
\item $\alpha :=\;$\emph{If things exist at a given point in time, then at
any given previous moment in time things must have existed.}

\item $\beta :=\;$\emph{Things exist today.}

\item $\gamma :=\;$\emph{The dawn of time is previous to all else.}

\item $\delta :=\;$\emph{Things existed at the dawn of time.}
\end{itemize}

In hybrid temporal logic we can express this argument quite simply:

Hypothesis:

\begin{itemize}
\item $\alpha := q \rightarrow [P]q$

\item $\beta := @_hq$

\item $\gamma := @_a[P]\perp$
\end{itemize}

Conclusion

\begin{itemize}
\item $\delta := @_a q$
\end{itemize}

To prove $\delta$ from the hypothesis we can use the trichotomy axiom of
temporal logic 
$$
@_a h\vee @_a \langle P\rangle h \vee @_h \langle P\rangle a
$$

