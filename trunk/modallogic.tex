
%==========================================================================
\section{Classical conception of modal logic}

Modal logics are obtained from classical standard logic systems
---propositional logic, first order logic, second order logic, etc.--- by
adding \emph{``non-truth-functional''}
operators, named modalities. By using these operators we gain the ability to
qualify truth.

\subsection{Qualified truth}

In \textit{Modal Logic,} truth can be qualified, that is to say, beyond its
descriptive or denotative content, expressions have also an auto-reflexive
dimension. Modal logicians created a formalism able to capture dynamic
instances, in which truth can be relative. The most frequently qualifiers
are: \emph{necessary, possible, contingent, obligatory, always, from now on,
after executing a program, an action, believed, known, etc. }In fact, there
is not a unique modal logic but a whole family of modal logics.

From a contemporary perspective, modal logics are:

\begin{enumerate}
\item Tools for modeling and proving. Many notions in linguistic,
philosophy, computing and science are not absolute but rather have a 
\emph{``modal''} character. Modal logic, thus,
provides a language and semantics to represent these notions in a rigorous
way, adding as well a reasoning mechanism to obtain deductions or
derivations. Modal systems has been used to reason about \emph{necessity}
and \emph{possibility}, \emph{time}, \emph{beliefs}, \emph{computation}, and
many other. All these applications share graph-like structures to represent
their main concepts (networks of possible worlds, flows of time, epistemic
alternatives, computational states, etc.)

\item Tools for identifying interesting fragments of classical logics (via
translation techniques)~\cite{Blackburnetal2007}. Modal logics (and
their equivalent fragment via translation into the standard logic) can be
seen as fragments of standard logics that inherit its standard semantics
(classical first order or higher order) but restricts expressive power (by
using operators instead of quantification semantics). However, modal logics
are systems with logical properties different from the original standard
system, such as the decidability property of many modal logics. In
particular, under this view, modal logics are interesting tools to study the
balance between expressiveness and computational complexity in formal systems.
\end{enumerate}

\subsection{The relational semantics impact}

The best known style of modal semantics is the \emph{relational} or 
\emph{Kripke} semantics. After studying the class of Kripke models and their
semantics, it is easy to reach the conclusion that pure modal logic axioms
capture semantical properties of the accessibility relation $R$, \emph{i.e.},
modal formulas are valid in models with the corresponding property of $R$:

\begin{itemize}
\item $T:=\square \varphi \rightarrow \varphi $ is valid, for reflexive $R$

\item $4:=\square \varphi \rightarrow \square \square \varphi \ $ is valid,
for transitive $R$

\item $D:=\square \varphi \rightarrow \Diamond \varphi $ is valid, for
serial $R$

\item $B:=\varphi \rightarrow \square \Diamond \varphi $ is valid, for
symmetric\ $R$

\item $5:=\Diamond \varphi \rightarrow \square \Diamond \varphi $ is valid,
for euclidean\ $R$

\item $G:=\Diamond \square \varphi \rightarrow \square \Diamond \varphi $ is
valid, for incestuous\ $R$
\end{itemize}

In fact, the axioms of a modal logic try to characterize the properties of
their own accessibility relation. The language is very effective, but not
always successful. Moreover, we have other alternatives languages. To put it
succinctly:

\begin{itemize}
\item Modal models are relational structures and they are extremely common
in \textit{Classical Model Theory}. 

\item On the other hand, temporal relation is \emph{asymmetric} and
\emph{irreflexive}, but these are not definable in orthodox tense logic.
\end{itemize}

\subsection{Evaluating modal logic}

Modal logics are a counterpart of classical logics. The question is what do
we gain when shifting from classical to modal logic? There are five key
points that make modal logics a very useful tool:

\begin{enumerate}
\item Better understanding, precision and useful axiomatization of the
structures formed by \emph{states, transitions} and \emph{procedures}.

\item Very concise formal expression (operator designated \emph{ad hoc})

\item Modal logics are \emph{decidable}, which is an improved situation in
comparison to the one encountered in FOL or other higher order logics.

\item \emph{Locally} focused: to better understand the \emph{structure of
transitions} between states we place ourselves inside the structure and
travel along

\item We gain an \emph{internal }perspective (AUTOMATON:= visiting the
accessible states)
\end{enumerate}

On the other hand, what is missing when shifting to modal logics?

\begin{enumerate}
\item States are crucial, but we cannot refer to them.

\item The accessibility relation is essential, but it is not explicit in the
language.

\item In tableau calculus metalanguage tools are needed to solve this
shortcoming.
\end{enumerate}

