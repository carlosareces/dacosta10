%==========================================================================
\section{Completeness}\label{completeness}

\subsection{Theoretical and practical relevance of completeness}

For a logic $\Delta$ and a suitable class of structures $\mathfrak{K}$,
completeness (weak) and soundness states%
$$
\vdash _{\CAL(\Delta)}\varphi \ \ \text{iff}\ \ \models _{\mathfrak{K}%
}\varphi
$$
That is, a formula $\varphi $ is a logical theorem of logic $\Delta$ (is
provable in a deductive calculus $\CAL(\Delta)$ for $\Delta$) if and only
if $\varphi$ is valid with respect of the class of structures $\mathfrak{K}$. That is, $\THEO_{\CAL(\Delta)} = \VAL_{\mathfrak{K}}$. Strong completeness
establishes 
$$
\Gamma \vdash _{\CAL(\Delta)}\varphi \ \ \text{iff}\ \ \Gamma 
\models _{\mathfrak{K}}\varphi
$$

\paragraph{Theoretical relevance:}

We can say that we don't know a logic till we haven't identified its set of
valid formulas. Intuitively, we can say that the \emph{logicality} of a
given formal language resides in the set $\VAL_{\mathfrak{K}}$ of valid
sentences. Each structure $\mathcal{A}\in \mathfrak{K}$ select from the set
of all sentences those which are true under this particular interpretation.
This set of formulas is usually called the theory of a structure 
$Th(\mathcal{A})$, and it is characteristic of each structure. But all such
theories share a common nucleus which is the set $\VAL_{\mathfrak{K}}$.
Because they are true in all structures, these sentences do not refer to
particular properties of a particular structure. Hence, $\VAL_{\mathfrak{K}}$
clearly does not characterize any particular structure.

Does this set characterize anything?

The answer is yes, it characterizes the logic in question itself. It
represents what the logic \emph{`has to say'} about any arbitrary
structure. If we are able to \emph{`generate'} this set easily, we would
have finally captured the \emph{essence} of a logic.

\paragraph{Practical relevance:}

The semantic notion of consequence perfectly defines when a formula $\varphi 
$ follows from a given set of formulas $\Gamma $ (it is `enough' to verify
that $\varphi $ is true in all models where $\Gamma $ is true); but it does
not provides for an \emph{`algorithm'} that helps us verify this relation
(how do we access the models where $\Gamma $ is true to see if $\varphi$ is
also true there). In principle, we need to check every single model. This is
when the set $\THEO_{\CAL(\Delta)}$ of theorems, and the notion of
completeness, come to our help. The set $\THEO_{\CAL(\Delta)}$ is specified
through a calculus that explicitly defines when a formula follows from
others: we can infer or deduce that $\varphi $ follows in the deductive
calculus using the set $\Gamma$ as hypothesis. In other words, we can
establish a chain of inference from the premises in $\Gamma$ to the
conclusion $\varphi$. Actually, this operational definition of consequence
seems even more adequate and closer to the intuitive notion of inference,
given that it reflects the discursive character of the process. The
completeness theorem just tells us that both definitions are equivalent.

In modal logic the classical techniques to prove completeness are: 
\emph{canonical models} and \emph{filtration}. Now, what about proving
completeness for type theory or other higher order systems?

\subsection{Completeness in the theory of types}

As we stated in the introduction, Henkin defined what he called 
\emph{general models} and proved the completeness theorem for type 
theory~\cite{Henkin1950}. He first proved the following theorem, linking
consistency and satisfiability in a general model.

\begin{theorem}
If $\Lambda$ is any consistent set
of cwffs there is a general model (in which its domain  $D_{\alpha}$
is denumerable) with respect to which $\Lambda$ is satisfiable.
\end{theorem}

The completeness theorem is obtained as a direct corollary.

The proof follows these key steps:

\begin{enumerate}
\item \emph{``\ldots to construct a maximal consistent set} $\Gamma$ 
\emph{such that} $\Gamma$ \emph{contains} $\Lambda$ \emph{\ldots''}. Maximal
consistent sets describe with enormous precision a possible model for
themselves.

\item \emph{``Two cwffs} $A_{\alpha}$ \emph{and} $B_{\alpha}$ 
\emph{of type} $\alpha$ \emph{will be called equivalent iff} $\Gamma
\vdash A_{\alpha }\leftrightarrow B_{\alpha }$\emph{''}. This is a genuine
congruence relation.

\item \emph{``We now define by induction on} $\alpha$ \emph{a 
frame\fixme{why ``frame of domains''} of domains} $\{D_\alpha\}$ \emph{and simultaneously a
one-to-one mapping} $\Phi$ \emph{of equivalence classes onto the domains} $D_{\alpha}$ 
\emph{such that} $\Phi (\left[ A_{\alpha }\right]$
\emph{is in} $D_{\alpha }$\emph{''}.
\end{enumerate}

\subsection{Completeness of the first-order functional calculus}

Leon Henkin also proved the completeness of the first-order 
calculus~\cite{Henkin1949} by using his new Henkin-proof style (a mix of semantics and
syntax put together)

\begin{theorem}
If $\Lambda$  is a set of formulas of 
$S_{0}$  in which no member has any occurrence of a free individual
variables, and if  $\Lambda$  is consistent, then  $\Lambda$
 is simultaneously satisfiable in a domain of individuals having
the same cardinal number as the set of primitive symbols of $S_{0}$.
\end{theorem}

The proof follows the following steps:

\begin{enumerate}
\item As in the previous proof for type theory a maximal consistent set
$\Gamma_{\omega}$ is build. \emph{``It is easy to see that} 
$\Gamma_{\omega}$ \emph{possesses the following properties:}

$\Gamma_{\omega}$ \emph{is a maximal consistent set of cwffs of}
$S_{\omega}$.

\emph{If a formula} $(\exists x)A$ \emph{is in} $\Gamma_{\omega}$
\emph{then} $\Gamma_{\omega}$ \emph{also contains a formula}
$\overline{A}$ \emph{\ldots by substituting some constant for each free
occurrence of the variable} $x$ \emph{''}.

\item An interpretation is constructed on top of this set

\emph{``In fact we take as our domain} $\Im$ \emph{simply the set of
individuals constants of} $S_{\omega}$, \emph{and we assign to each such
constant\ldots itself\ldots as denotation.''} or \emph{``If we take the
domain} $\overline{\Im }$ \emph{of equivalence classes\ldots ''}.
\end{enumerate}

\subsection{The completeness of hybrid type theory (HTT)}

We present here the completeness of $\mathcal{HTT}$ following Henkin's
conception.

\begin{itemize}
\item $\mathcal{HTT}$ should be incomplete (with respect to standard models)
due to its expressive power, but we know from \emph{Completeness in the
theory of types} the use \emph{general models}. The main notions will be:

\begin{itemize}
\item A \emph{pre-structure} is a structure $\mathcal{M}$
verifying all the conditions for a standard structure, except for the
fullness condition on the domains of the hierarchy of types; it is only
required that $\mathsf{D}_{\langle a,b\rangle }\subseteq 
\mathsf{D}_{b}{}^{\mathsf{D}_{a}}$

\item A \emph{general structure for} $\mathcal{HTT}$ is a
pre-structure closed under interpretation, that is, for any meaningful
expression in $\mathsf{ME}_{a}$, its interpretation is a member of 
$\mathsf{D}_{a}$.
\end{itemize}

\item Being $\mathcal{HTT}$ modal we can doubt about the method of proof,
but we learned from \emph{The completeness of the First-Order Functional
Calculus} the use of \emph{constants}.

\begin{itemize}
\item In $\mathcal{HTT}$ we use \emph{rigid terms}\textbf{\ ---}the set of 
$\mathsf{RIGIDS}$ is defined inductively---

\item Moreover, we have this theorem:\ 
Let $\langle \mathcal{M},g\rangle $ be a model, if $\gamma \in \mathsf{RIGIDS}$ then $[[\gamma]]^{\mathcal{M},w,g}=[[\gamma ]]^{\mathcal{M},v,g}$ for all $w,v\in W$
\end{itemize}
\end{itemize}

We are also loyal to Henkin's construction:

\begin{enumerate}
\item By defining an \emph{extension to a maximal consistent set adding
more conditions.}

If $\Sigma$ is any consistent set of formulas, there is an extension to set 
$\Sigma ^{\omega}$ named, $\Diamond$-saturated, $\exists$-saturated,
maximal consistent set.

\item By defining an \emph{equivalent relation using the extended set.

Given $\Gamma^{\ast}$, a maximal consistent set which is named, $\Diamond$
saturated and $\exists$ saturated, we will define an equivalence relation
for each type, using derivability of equality from $\Gamma^{\ast}$. 

$\alpha _{a}\approx \beta _{a}$ iff $\alpha _{a}=\beta _{a}\in 
\Gamma ^{\ast}$ for all $\alpha _{a}$, $\beta _{a}\in \mathsf{RIGIDS}_{a}$ where $a\in 
\mathsf{TYPES-}\left\{ t\right\} $

\item By introducing the \emph{definition of the general structure.}

\begin{enumerate}
\item We define a general structure $\mathcal{M}=\langle \mathcal{S},\mathsf{%
F}\rangle $ having a skeleton 
\begin{equation*}
\mathcal{S}=\left\langle \langle \mathsf{D}_{a}\rangle _{a\in \mathsf{TYPES}%
},W,R\right\rangle
\end{equation*}%
and a denotation function $\mathsf{F}$.

\item The definition of the hierarchy $\langle \mathsf{D}_{a}\rangle _{a\in 
\mathsf{TYPES}}$ is done by induction on $a\in \mathsf{TYPES}$ and
simultaneously we define a one-to-one mapping $\Phi $ of equivalent classes
onto the domains such that $\Phi (\left[ \alpha _{a}\right] )\in \mathsf{D}%
_{a}$ for all rigid expression $\alpha _{a}$ of type $a$. After defining the
hierarchy and the mapping we have to prove that $\Phi $ is:

\begin{enumerate}
\item well defined, one-to-one, onto

\item we set
\end{enumerate}

\item For type $ba$ 
\begin{equation*}
\begin{tabular}{llll}
$\Phi :$ & $\left\{ \left[ \gamma _{ba}\right] \mid \gamma _{ba}\in \mathsf{%
RIGIDS}_{ba}\right\} $ & $\Longrightarrow $ & $\mathsf{D}_{ba}$ \\ 
& \multicolumn{1}{r}{$\left[ \gamma _{ba}\right] $} & $\longmapsto $ & $\Phi
(\left[ \gamma _{ba}\right] )\in \mathsf{D}_{ba}$%
\end{tabular}%
\end{equation*}%
where%
\begin{equation*}
\begin{tabular}{lrll}
& $\Phi (\left[ \gamma _{ba}\right] ):\mathsf{D}_{b}$ & $\Longrightarrow $ & 
$\mathsf{D}_{a}$ \\ 
& $\Phi (\left[ \beta _{b}\right] )$ & $\longmapsto $ & $\Phi (\left[ \gamma
_{ba}\beta _{b}\right] )\in \mathsf{D}_{a}$%
\end{tabular}%
\end{equation*}
\end{enumerate}
\end{enumerate}

Now the proof can be done by following the Henkin construction step by step.


